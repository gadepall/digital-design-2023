\documentclass{article}
\usepackage{multirow}
\usepackage{blindtext}
\usepackage{amsmath}
\usepackage{capt-of}
\usepackage{circuitikz}
\usepackage{listings}
\usepackage{./karnaugh-map}
\usetikzlibrary{shapes.geometric}

\lstset{
	language=C++,
	basicstyle=\ttfamily\footnotesize,
	breaklines=true,
	frame=lines
}

\title{Implementation of Boolean Logic in Arduino using IC 7474}
\date{February 2023}
\author{Sai Harshith Kalithkar\\harshith.work@gmail.com\\FWC22118\\IIT Hyderabad-Future Wireless Communication Assignment}

\begin{document}
\maketitle
	\tableofcontents
\pagebreak
\section{Problem}
	(GATE EC-2022)\\
	Q.43. For the circuit shown, the clock frequency is $f_0$ and the duty cycle is $25 \%$. For the signal at the $Q$ output of the Flip-Flop,
\\
\begin{figure}[h]
\begin{figure}[!ht]
\centering
\resizebox{1\textwidth}{!}{%
\begin{circuitikz}
\tikzstyle{every node}=[font=\Large]

\draw [, line width=0.9pt](4,6) to[short] (17,6);
\draw (4,6) to[short] (17,6);
\draw [, line width=0.9pt ] (5,7) rectangle (9,11);
\draw [, line width=0.9pt ] (15,7) rectangle (19,11);
\draw (11,10) to[short] (11.5,10);
\draw (11,9.5) to[short] (11.5,9.5);
\draw (11.5,10) node[ieeestd nand port, anchor=in 1, scale=0.89](port){} (port.out) to[short] (13.5,9.75);
\draw [, line width=0.9pt](11,9.5) to[short] (11,8.75);
\draw [, line width=0.9pt](9,10) to[short] (11,10);
\draw [, line width=0.9pt](13.5,9.75) to[short] (14,9.75);
\draw [, line width=0.9pt](14,9.75) to[short] (14,10);
\draw [, line width=0.9pt](14,10) to[short] (15,10);
\draw [, line width=0.9pt](19,10) to[short] (20.5,10);
\draw [, line width=0.9pt](20,10) to[short] (20,12);
\draw [, line width=0.9pt](20,12) to[short] (20,13);
\draw[, line width=0.9pt] (20,13) to[short] (10,13);
\draw [, line width=0.9pt](10,10) to[short] (10,12);
\draw (9,13) to[short] (8.75,13);
\draw (9,12.5) to[short] (8.75,12.5);
\draw (8.75,13) node[ieeestd xor port, anchor=in 2, scale=0.89, rotate=180](port){} (port.out) to[short] (6.75,12.75);
\draw[, line width=0.9pt] (10,13) to[short] (8.75,13);
\draw [, line width=0.9pt](10,12) to[short] (10,12.5);
\draw[, line width=0.9pt] (10,12.5) to[short] (9,12.5);
\draw[, line width=0.9pt] (6.75,12.75) to[short] (4,12.75);
\draw [, line width=0.9pt](4,12.75) to[short] (4,10);
\draw [, line width=0.9pt](4,10) to[short] (5,10);
\node [font=\Large] at (7,9) {A};
\node [font=\Large] at (17,9) {B};
\node [font=\Large] at (5.25,10) {D};
\node [font=\Large] at (15.25,10) {D};
\node [font=\Large] at (18.5,10) {Q};
\node [font=\Large] at (8.5,10) {Q};
\node [font=\Large] at (8.5,8) {$\bar Q$};
\node [font=\Large] at (18.5,8) {$\bar Q$};
\node [font=\Large] at (6,7.75) {CK};
\node [font=\Large] at (16,8) {CK};
\node [font=\Large] at (3.25,6) {CLK};
\node [font=\Large] at (11,8.5) {Xin};
\draw [, line width=0.9pt](6.75,6) to[short] (6.75,7);
\draw [, line width=0.9pt](17,6) to[short] (17,7);
\draw [line width=0.9pt, short] (6.25,7) -- (6.75,7.75);
\draw [line width=0.9pt, short] (6.75,7.75) -- (7.25,7);
\draw [line width=0.9pt, short] (16.5,7) -- (17,7.75);
\draw [line width=0.9pt, short] (17,7.75) -- (17.5,7);
\node [font=\Large] at (11.5,10.5) {$Q_A$};
\node [font=\Large] at (20.25,9.5) {$Q_B$};
\end{circuitikz}
}%

\label{fig:ide/fsm/figs/circuit}
\end{figure}

	\caption{Diagram}
	\label{fig:1}
\end{figure}

\begin{enumerate}
	\item frequency of $\frac{f_0}{4}$ and duty cycle is 50$\%$
	\item frequency of $\frac{f_0}{4}$ and duty cycle is 25$\%$
	\item frequency of $\frac{f_0}{2}$ and duty cycle is 50$\%$
	\item frequency of $f_0$ and duty cycle is 25$\%$ \\
\end{enumerate}

\section{Introduction}
		The Aim is to implement the above circuit in Arduino using IC 7474. IC 7474 is a dual positive-edge-triggered D-type flip-flop, which means it has two seperate flip-flop that are triggered by the rising edge of a clock signal. A 2-bit binary counter can be implemented using 2 D Flip-flops similarly a JK Flip-flop can be implemented using one D Flip-flop. Thus we will use two IC 7474 to implement the whole circuit.\\

		The LSB output of the 2-bit binary counter is given to J and K inputs of the JK Flip-flop which then gives the final Q output of the circuit. Since the inputs given to J and K are same it acts as T Flip-flop.\\
\section{Components}
	\begin{enumerate}
		\item Arduino UNO
		\item IC 7474 - 2 units
		\item Breadboard
		\item Jumper Wires (M-M) \\
	\end{enumerate}

\section{Hardware}
	The IC 7474 is a type of flip-flop integrated circuit that is commonly used indigital electronics applications. It is a dual positive-edge-triggered by the rising edge of a clock signal. Below is the pin diagram of IC 7474. \\
	\begin{figure}[h]
		\centering
	\input{figs/pinDiagram.tex}
		\caption{7474 Pin Diagram}
		\label{fig:2}
	\end{figure}

	The connections between Arduino UNO and two IC 7474 is given in below Table \\
	\begin{table}[h]
	\begin{center}
	\begin{tabular}{|c|c|c|c|c|c|c|c|c|}
        \hline & INPUT & \multicolumn{3}{|c|}{OUTPUT} & \multicolumn{2}{|c|}{CLOCK} & Vcc & GND \\
        \hline ARDUINO & D6 & D3 & D4 & D5 & \multicolumn{2}{|c|}{D2} & 5V & GND \\
        \hline 7447 && 5 & 9 && 3 & 11 & 14 & 7 \\
        \hline 7474 & 5 &&& 2 & \multicolumn{2}{|c|}{3} & 14 & 7 \\
        \hline
    \end{tabular}

	\end{center}
		\caption{Connections}
		\label{table:1}
	\end{table}

	The truth table for the circuit is given in below table \\
	\begin{table}[h]
		\begin{center}
	\begin{tabular}{|c|c|c|c|c|c|c|}
        \hline counter & MSB & LSB & J & K & Q(t) & Q(t+1) \\
        \hline 0 & 0 & 0 & 0 & 0 & 0 & 0 \\
        \hline 1 & 0 & 1 & 1 & 1 & 0 & 1 \\
        \hline 2 & 1 & 0 & 0 & 0 & 1 & 1 \\
        \hline 3 & 1 & 1 & 1 & 1 & 1 & 0 \\
        \hline
        \end{tabular}

		\caption{Truth Table}
		\label{table:2}
		\end{center}
	\end{table}
	
	The Kmap for the circuit is \\
	\begin{figure}[h]
		\centering
	\begin{karnaugh-map}[4][4][2][$T$][$S$][$R$][$Q$][$P$]                                                   
	\minterms{6,14,22,30,31}                                                                        	
	\autoterms[0]                                                                                            
	\implicant{6}{14}                                                                                        
	\implicant{22}{30}                                                                                       
	\implicant{31}{30}                                                                               
\end{karnaugh-map}

	\caption{Kmap}
	\label{fig:3}
	\end{figure}

\section{Software}
	The Arduino code for the given circuit using IC 7474 is \\
	\lstinputlisting{gate2022.cpp}
\end{document}
