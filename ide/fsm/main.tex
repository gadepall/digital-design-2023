\iffalse
\documentclass[journal,2pt,twocolumn]{IEEEtran}
%
\usepackage{setspace}
\usepackage{gensymb}
\usepackage{xcolor}
\usepackage{caption}
\usepackage[hyphens,spaces,obeyspaces]{url}
%\usepackage{subcaption}
%\doublespacing
\singlespacing

%\usepackage{graphicx}
%\usepackage{amssymb}
%\usepackage{relsize}
\usepackage[cmex0]{amsmath}
\usepackage{mathtools}
%\usepackage{amsthm}
%\interdisplaylinepenalty=2500
%\savesymbol{iint}
%\usepackage{txfonts}
%\restoresymbol{TXF}{iint}
%\usepackage{wasysym}
\usepackage{amsthm}
\usepackage{mathrsfs}
\usepackage{txfonts}
\usepackage{stfloats}
\usepackage{cite}
\usepackage{cases}
\usepackage{subfig}
%\usepackage{xtab}
\usepackage{longtable}
\usepackage{multirow}
%\usepackage{algorithm}
%\usepackage{algpseudocode}
\usepackage{enumerate}
\usepackage{mathtools}
\usepackage{eenrc}
%\usepackage[framemethod=tikz]{mdframed}
\usepackage[breaklinks]{hyperref}
%\usepackage{breakcites}
\usepackage{listings}
    \usepackage[latin]{inputenc}                                 %%
    \usepackage{color}                                            %%
    \usepackage{array}                                            %%
    \usepackage{longtable}                                        %%
    \usepackage{calc}                                             %%
    \usepackage{multirow}                                         %%
    \usepackage{hhline}                                           %%
    \usepackage{ifthen}                                           %%
  %optionally (for landscape tables embedded in another document): %%
    \usepackage{lscape}     

\usepackage{tikz}
\usepackage{circuitikz}
\usepackage{karnaugh-map}
\usepackage{pgf}
\usepackage[hyphenbreaks]{breakurl}

%\usepackage{url}
%\def\UrlBreaks{\do\/\do-}





%\usepackage{stmaryrd}


%\usepackage{wasysym}
%\newcounter{MYtempeqncnt}
\DeclareMathOperator*{\Res}{Res}
%\renewcommand{\baselinestretch}{2}
\renewcommand\thesection{\arabic{section}}
\renewcommand\thesubsection{\thesection\arabic{subsection}}
\renewcommand\thesubsubsection{\thesubsection\arabic{subsubsection}}

\renewcommand\thesectiondis{\arabic{section}}
\renewcommand\thesubsectiondis{\thesectiondis\arabic{subsection}}
\renewcommand\thesubsubsectiondis{\thesubsectiondis\arabic{subsubsection}}

% correct bad hyphenation here
\hyphenation{op-tical net-works semi-conduc-tor}

%\lstset{
%language=C,
%frame=single, 
%breaklines=true
%}

%\lstset{
	%%basicstyle=\small\ttfamily\bfseries,
	%%numberstyle=\small\ttfamily,
	%language=Octave,
	%backgroundcolor=\color{white},
	%%frame=single,
	%%keywordstyle=\bfseries,
	%%breaklines=true,
	%%showstringspaces=false,
	%%xleftmargin=-0mm,
	%%aboveskip=-mm,
	%%belowskip=0mm
%}

%\surroundwithmdframed[width=\columnwidth]{lstlisting}
\def\inputGnumericTable{}                                 %%
\lstset{
%language=C,
frame=single, 
breaklines=true,
columns=fullflexible
}
 

\begin{document}
%

\theoremstyle{definition}
\newtheorem{theorem}{Theorem}[section]
\newtheorem{problem}{Problem}
\newtheorem{proposition}{Proposition}[section]
\newtheorem{lemma}{Lemma}[section]
\newtheorem{corollary}[theorem]{Corollary}
\newtheorem{example}{Example}[section]
\newtheorem{definition}{Definition}[section]
%\newtheorem{algorithm}{Algorithm}[section]
%\newtheorem{cor}{Corollary}
\newcommand{\BEQA}{\begin{eqnarray}}
\newcommand{\EEQA}{\end{eqnarray}}
\newcommand{\define}{\stackrel{\triangle}{=}}

\bibliographystyle{IEEEtran}
%\bibliographystyle{ieeetr}

\providecommand{\nCr}[2]{\,^{#}C_{#2}} % nCr
\providecommand{\nPr}[2]{\,^{#}P_{#2}} % nPr
\providecommand{\mbf}{\mathbf}
\providecommand{\pr}{\ensuremath{\Pr\left(#\right)}}
\providecommand{\qfunc}{\ensuremath{Q\left(#\right)}}
\providecommand{\sbrak}{\ensuremath{{}\left[#\right]}}
\providecommand{\lsbrak}{\ensuremath{{}\left[#\right}}
\providecommand{\rsbrak}{\ensuremath{{}\left#\right]}}
\providecommand{\brak}{\ensuremath{\left(#\right)}}
\providecommand{\lbrak}{\ensuremath{\left(#\right}}
\providecommand{\rbrak}{\ensuremath{\left#\right)}}
\providecommand{\cbrak}{\ensuremath{\left\{#\right\}}}
\providecommand{\lcbrak}{\ensuremath{\left\{#\right}}
\providecommand{\rcbrak}{\ensuremath{\left#\right\}}}
\providecommand{\ceil}{\left \lceil # \right \rceil }
\theoremstyle{remark}
\newtheorem{rem}{Remark}
\newcommand{\sgn}{\mathop{\mathrm{sgn}}}
\providecommand{\abs}{\left\vert#\right\vert}
\providecommand{\res}{\Res\displaylimits_{#}} 
\providecommand{\norm}{\lVert#\rVert}
\providecommand{\mtx}{\mathbf{#}}
\providecommand{\mean}{E\left[ # \right]}
\providecommand{\fourier}{\overset{\mathcal{F}}{ \rightleftharpoons}}
%\providecommand{\hilbert}{\overset{\mathcal{H}}{ \rightleftharpoons}}
\providecommand{\system}{\overset{\mathcal{H}}{ \longleftrightarrow}}
	%\newcommand{\solution}[2]{\textbf{Solution:}{#}}
\newcommand{\solution}{\noindent \textbf{Solution: }}
\providecommand{\dec}[2]{\ensuremath{\overset{#}{\underset{#2}{\gtrless}}}}
%\numberwithin{equation}{subsection}
\numberwithin{equation}{section}
%\numberwithin{problem}{subsection}
%\numberwithin{definition}{subsection}
\makeatletter
\@addtoreset{figure}{problem}
\makeatother

\let\StandardTheFigure\thefigure
%\renewcommand{\thefigure}{\theproblem\arabic{figure}}
\renewcommand{\thefigure}{\theproblem}


%\numberwithin{figure}{subsection}

%\numberwithin{equation}{subsection}
%\numberwithin{equation}{section}
%%\numberwithin{equation}{problem}
%%\numberwithin{problem}{subsection}
\numberwithin{problem}{section}
%%\numberwithin{definition}{subsection}
%\makeatletter
%\@addtoreset{figure}{problem}
%\makeatother
\makeatletter
\@addtoreset{table}{problem}
\makeatother

\let\StandardTheFigure\thefigure
\let\StandardTheTable\thetable
%%\renewcommand{\thefigure}{\theproblem\arabic{figure}}
%\renewcommand{\thefigure}{\theproblem}
\renewcommand{\thetable}{\theproblem}
%%\numberwithin{figure}{section}

%%\numberwithin{figure}{subsection}

\vspace{3cm}

\title{ 
	\logo{
Finite State Machine
	}
}



% paper title
% can use linebreaks \\ within to get better formatting as desired
%\title{Matrix Analysis through Octave}
%
%
% author names and IEEE memberships
% note positions of commas and nonbreaking spaces ( ~ ) LaTeX will not break
% a structure at a ~ so this keeps an author's name from being broken across
% two lines
% use \thanks{} to gain access to the first footnote area
% a separate \thanks must be used for each paragraph as LaTeX2e's \thanks
% was not built to handle multiple paragraphs
%

\author{G V V Sharma$^{*}$% <-this % stops a space
\thanks{*The author is with the Department
of Electrical Engineering, Indian Institute of Technology, Hyderabad
502285 India e-mail:  gadepall@iithacin All content in this manual is released under GNU GPL  Free and open source}% <-this % stops a space
%\thanks{J Doe and J Doe are with Anonymous University}% <-this % stops a space
%\thanks{Manuscript received April 9, 2005; revised January , 2007}}
}
% note the % following the last \IEEEmembership and also \thanks - 
% these prevent an unwanted space from occurring between the last author name
% and the end of the author line ie, if you had this:
% 
% \author{lastname \thanks{} \thanks{} }
%                     ^------------^------------^----Do not want these spaces!
%
% a space would be appended to the last name and could cause every name on that
% line to be shifted left slightly This is one of those "LaTeX things" For
% instance, "\textbf{A} \textbf{B}" will typeset as "A B" not "AB" To get
% "AB" then you have to do: "\textbf{A}\textbf{B}"
% \thanks is no different in this regard, so shield the last } of each \thanks
% that ends a line with a % and do not let a space in before the next \thanks
% Spaces after \IEEEmembership other than the last one are OK (and needed) as
% you are supposed to have spaces between the names For what it is worth,
% this is a minor point as most people would not even notice if the said evil
% space somehow managed to creep in



% The paper headers
%\markboth{Journal of \LaTeX\ Class Files,~Vol~6, No~, January~2007}%
%{Shell \MakeLowercase{\textit{et al}}: Bare Demo of IEEEtrancls for Journals}
% The only time the second header will appear is for the odd numbered pages
% after the title page when using the twoside option
% 
% *** Note that you probably will NOT want to include the author's ***
% *** name in the headers of peer review papers                   ***
% You can use \ifCLASSOPTIONpeerreview for conditional compilation here if
% you desire




% If you want to put a publisher's ID mark on the page you can do it like
% this:
%\IEEEpubid{0000--0000/00\$0000~\copyright~2007 IEEE}
% Remember, if you use this you must call \IEEEpubidadjcol in the second
% column for its text to clear the IEEEpubid mark



% make the title area
\maketitle

\tableofcontents

\bigskip

\renewcommand{\thefigure}{\theenumi}
\renewcommand{\thetable}{\theenumi}


\begin{abstract}
%\boldmath
	\fi
We explain  a state machine by deconstructing the decade counter

\section{The Decade Counter}
The block diagram of a decade counter (repeatedly counts up from 0 to 9)
is available in Fig \ref{fig:decade_counter}  The {\em incrementing } decoder
and {\em display} decoder are part of {\em combinational} logic, while
the {\em delay} is part of {\em sequential} logic
\iffalse
\begin{figure}[!h]
\resizebox {\columnwidth} {!} {
\input{ide/fsm/figs/decade_counter}
}
\caption{The decade counter}
\label{fig:decade_counter}
\end{figure}
\fi
%
\section{Finite State Machine}
%
\begin{enumerate}

\item Fig \ref{fig:fsm_counter} shows a {\em finite state machine} (FSM) diagram for the decade counter in Fig \ref{fig:decade_counter}  $s_0$ is the state when the input to the incrementing decoder is 0  The {\em state transition table} for the FSM is Table \ref{tab:ide/7447/counter_decoder},
%	0 in \cite{gvv_kmap}
		where the present state is denoted by the variables $W,X,Y,Z$ and the next state by $A,B,C,D$.  
\begin{figure}[!h]
\centering
%\resizebox {\columnwidth} {!} {
\input{ide/fsm/figs/fsm_counter}
%}
\caption{FSM for the decade counter}
\label{fig:fsm_counter}
\end{figure}
\item The FSM implementation is available in Fig \ref{fig:dff}  The {\em flip-flops} hold the input for the time that is given by the {\em clock}  This is nothing but the implementation of the {\em Delay} block in Fig \ref{fig:decade_counter}
%
\begin{figure}[!h]
\resizebox {\columnwidth} {!} {
\input{ide/fsm/figs/dff}
}
\caption{Decade counter FSM implementation using D-Flip Flops}
\label{fig:dff}
\end{figure}
%
\item The hardware cost of the system is given by
\begin{equation}
\text{No of D Flip-Flops} = \myceil{\log_{2}\brak{\text{No of States}}}
\end{equation}
For the FSM in Fig \ref{fig:fsm_counter}, the number of states is 9, hence the number flipflops required = 4  
\item Draw the state transition diagram for 
a decade down counter (counts from 9 to 0 repeatedly) using an FSM  
\item Write the state transition table for the down counter
\item Obtain the state transition equations with and without don't cares
\item Verify your design using an arduino
%\item Repeat the above exercises by designing a circuit that can detect 3 consecutive s in a bitstream 
\end{enumerate}


