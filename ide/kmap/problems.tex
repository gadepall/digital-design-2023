\begin{enumerate}
	\item 		Obtain the Minimal Form for the Boolean Expression
\label{prob:2013/d/6/d}
\hfill (CBSE 2013)
		\begin{align}
\label{eq:2013/d/6/d}
H(P,Q,R,S)=\sum(0,1,2,3,5,7,8,9,10,14,15)
		\end{align}
	\item Write the POS form for the function G shown in Table
\ref{tab:2013/c/6/d}.
\label{prob:2013/c/6/d}
\hfill (CBSE 2013)

		\begin{table}[!ht]
			\centering
		\begin{tabular}{ |c |c |c |c |}
 \hline
 U  &  V  &  W  &  G\\
 \hline
 0  &  0  &  0  &  1\\
 \hline
 0  &  0  &  1  &  0\\
 \hline
 0  &  1  &  0  &  1\\
 \hline
 0  &  1  &  1  &  0\\
 \hline
 1  &  0  &  0  &  1\\
 \hline
 1  &  0  &  1  &  0\\
 \hline
 1  &  1  &  0  &  0\\
 \hline
 1  &  1  &  1  &  1\\
 \hline
 \end{tabular}
			\caption{}
\label{tab:2013/c/6/d}
 \end{table}
	\item Reduce the following Boolean Expression to its simplest form using K-Map 
\label{prob:2015-1/c/6/d}
\hfill (CBSE 2015)
		\begin{align}
\label{eq:2015-1/c/6/d}
			F(X,Y,Z,W)=(0,1,4,5,6,7,8,9,11,15)
		\end{align}
	\item 
		Derive a Canonical POS expression for a Boolean function F, represented by the following truth table 
\label{prob:2015-1/c/6/c}
\hfill (CBSE 2015)
		\begin{table}[!ht]
			\centering
		\begin{tabular}{|c|c|c|c|}
	\hline
X & Y & Z & F \\
\hline
0 & 0 & 0 & 1 \\  
\hline
0 & 0 & 1 & 0 \\ 
\hline
0 & 1 & 0 & 0 \\
\hline
0 & 1 & 1 & 1 \\
\hline
1 & 0 & 0 & 1 \\  
\hline
1 & 0 & 1 & 0 \\ 
\hline
1 & 1 & 0 & 0 \\
\hline
1 & 1 & 1 & 1 \\
\hline
\end{tabular}
		\caption{}
\label{tab:2015-1/c/6/c}
\end{table}
	\item 
\label{prob:2015/c/6/d}
\hfill (CBSE 2015)
		Reduce the following Boolean Expression to its simplest form using K-map
		\begin{align}
\label{eq:2015/c/6/d}
F(X,Y,Z,W)= \sum (0,1,6,8,9,10,11,12,15)
		\end{align}
	\item Reduce the following Boolean Expression to its simplest form using K-map.
\label{prob:2016/c/6/d}
\hfill (CBSE 2016)
		\begin{align}
\label{eq:2016/c/6/d}
			F(X,Y,Z,W)= \sum(2,6,7,8,9,10,11,13,14,15)
		\end{align}
	\item Derive a Canonical POS expression for a Boolean function  F, represented in Table
\ref{tab:2016/c/6/c}
\hfill (CBSE 2016)
\label{prob:2016/c/6/c}
		\begin{table}[!ht]
			\centering
			\begin{tabular}{|c|c|c|c|}
	\iffalse
	{0.8\textwidth} { 
  | >{\centering\arraybackslash}X 
  | >{\centering\arraybackslash}X 
  | >{\centering\arraybackslash}X | 
  | >{\centering\arraybackslash}X |
  }
  \fi
 \hline
 P & Q & R & F(P,  Q,  R) \\
 \hline
 0  & 0  & 0 & 0  \\
\hline
0  & 0  & 1 & 1  \\
\hline
0  & 1  & 0 & 1  \\
\hline
0  & 1  & 1 & 0  \\
\hline
1  & 0  & 0 & 0  \\
\hline
1  & 0  & 1 & 0  \\
\hline
1  & 1  & 0 & 1  \\
\hline
1  & 1  & 1 & 1  \\
\hline
\end{tabular}
			\caption{}
\label{tab:2016/c/6/c}
		\end{table}
	\item Verify the following 
\hfill (CBSE 2016)
\label{prob:2016/c/6/a}
		\begin{align}
\label{eq:2016/c/6/a}
A'+B'C = A'B'C' + A'BC' + A'BC + A'B'C + AB'C
		\end{align}


	\item Reduce the following boolean expression to it's simplest form using K-Map
\hfill (CBSE 2017)
\label{prob:2017-1/c/6/d}
		\begin{align}
F(X,Y,Z,W) = \sum(0,1,2,3,4,5,10,11,14)
\label{eq:2017-1/c/6/d}
		\end{align}
	\item
		Reduce the following Boolean Expression to its simplest form using K-Map.
\hfill (CBSE 2017)
\label{prob:2017/c/6/d}
		\begin{align}
E(U,V,Z,W)=   (2 , 3 , 6 , 8 , 9 , 10 , 11 , 12 , 13 )
\label{eq:2017/c/6/d}
		\end{align}
	\item Derive a canonical POS expression for a Boolean function $G$, represented by Table 
\ref{tab:2017/c/6/c}
\label{prob:2017/c/6/c}
\hfill (CBSE 2017)
		\begin{table}[htbp]
 \begin{center}
    \begin{tabular}{|l|c|c|c|c|c|c|c|c|} \hline 
  \textbf{X}& \textbf{Y} & \textbf{Z} &\textbf{G(X,Y,Z)} \\
 \hline
 0&0&0&0\\ \hline
0&0&1&0 \\ \hline
0&1&0&1\\ \hline
0&1&1&0  \\ \hline
1&0&0&1\\ \hline
1&0&1&1\\ \hline
1&1&0&0\\ \hline
1&1&1&1\\ \hline
\end{tabular}   
\end{center}
\caption{}
\label{tab:2017/c/6/c}
\end{table}
\item Derive a canonical POS expression for a Boolean function $FN$, represented by Table 
\ref{tab:2018/c/6/c}.
\label{prob:2018/c/6/c}
\hfill (CBSE 2018)
		\begin{table}[!ht]
			\centering
\begin{tabular}{|l|c|r|l|c|}
    \hline % <-- Alignments: 1st column left, 2nd middle and 3rd right, with vertical lines in between
      \textbf{X} & \textbf{Y} & \textbf{Z} & \textbf{FN(X,Y,Z)}\\
      \hline
      0 & 0 & 0 & 1\\
\hline
      0 & 0 & 1 & 1\\
\hline
      0 & 1 & 0 & 0\\
\hline
      0 & 1 & 1 & 0\\
\hline
      1 & 0 & 0 & 1\\
\hline
      1 & 0 & 1 & 0\\
\hline
      1 & 1 & 0 & 0\\
\hline
      1 & 1 & 1 & 1\\
      \hline      
   \end{tabular}
\caption{}
\label{tab:2018/c/6/c}
   \end{table}

\item 
	Reduce the following Boolean expression in the simplest form using K-Map.
		\begin{align}
F(P,Q,R,S) = \sum (0,1,2,3,5,6,7,10,14,15)
		\end{align}
\hfill (CBSE 2019)
\label{prob:2019/c/6/d}
%
\item Fig. \ref{fig:2020/gate/ec/10} below shows a muliplexer where S0 and S1 are the select lines, I0 to I3 are the input lines, EN is the enable line and F(P,Q,R) is the output. Find the boolean expression for output F as function of inputs P,Q,R using K-map. 
%
\label{prob:2020/gate/ec/10}
\hfill (GATE EC 2020)
\begin{figure}[ht]
\centering
	\includegraphics[width=1\columnwidth]{figs/2020-gate-ec-10.png}
\caption{}
\label{fig:2020/gate/ec/10}
\end{figure}
%
\item
	The four variable function $f$ is given in terms of min-terms as
		\begin{align}
	    f(A,B,C,D) = \sum m(2,3,8,10,11,12,14,15)
\label{eq:1991/gate/ec/9}
		\end{align}
	    Using the K-map minimize the function in the sum of products form. 
\label{prob:1991/gate/ec/9}
\hfill (GATE EC 1991)
\item Find the logic realized by the circuit in Fig. 
\ref{fig:1992/gate/ec/1/22}.
\label{prob:1992/gate/ec/1/22}
\hfill (GATE EC 1992)
\begin{figure}[ht]
\centering
	\includegraphics[width=1\columnwidth]{figs/1992-gate-ec-1-22.png}
\caption{}
\label{fig:1992/gate/ec/1/22}
\end{figure}
\item
	A combinational circuit has three inputs A, B and C and an output F. F is true only for the following input combinations. 
\hfill (GATE EC 1992)
\label{prob:1992/gate/ec/2/9}
	\begin{enumerate}
		\item A is false and B is true 
		\item A is false and C is true 
		\item A, B and C are all false 
		\item A, B and C are all true 
	\end{enumerate}
	\begin{enumerate}
\item Write the truth table for F. use the convention, true = 1 and false = 0. 
\item Write the simplified expression for F as a Sum of Products. 
\item Write the simplified expression for F as a product of Sums.
\end{enumerate}
\item Draw the logic circuit for Table 
\ref{tab:1993/gate/ec/5/7} using only NOR gates.
\label{prob:1993/gate/ec/5/7}
\hfill (GATE EC 1993)
	\begin{table}[!ht]
		\centering
		\begin{tabular}{|c|c|c|c|}
\hline
\textbf{C} &\textbf{B} & \textbf{A} & \textbf{Y} \\
\hline
0 & 0 & 0 & 1 \\  
\hline
0 & 0 & 1 & 1 \\ 
\hline
0 & 1 & 0 & 1 \\
\hline
0 & 1 & 1 & 0 \\
\hline
1 & 0 & 0 & 1 \\  
\hline
1 & 0 & 1 & 0 \\ 
\hline
1 & 1 & 0 & 0 \\
\hline
1 & 1 & 1& 0\\
\hline
\end{tabular}
\caption{}
\label{tab:1993/gate/ec/5/7}
\end{table}
\item
	Implement the following Boolean function in a 8x1 multiplexer.
\label{prob:1993/gate/ec/14}
\hfill (GATE EC 1993)
		\begin{align}
\label{eq:1993/gate/ec/14}
 Q = BC + ABD' + A'C'D  
		\end{align}
	\item Minimize the following Boolean function in 
\ref{eq:1999/gate/ec/2/10}.
\label{prob:1999/gate/ec/2/10}
		\begin{equation}
\label{eq:1999/gate/ec/2/10}
F= A'B'C'+A'BC'+A'BC+ABC'
\end{equation}
%
\item Find the Boolean expression for Table
\ref{tab:2005-gate-ec-54}.
\label{prob:2005-gate-ec-54}
\hfill (GATE EC 2005)
	\begin{table}[htbp]
		\centering
    \begin{tabular}{|l|c|c|c|c|c|c|c|c} \hline \textbf{A}
  & \textbf{B} & \textbf{C} & \textbf{X} \\
 \hline
        0&0&0&0 \\
        \hline
        0&0&1&0 \\
        \hline
        0&1&0&0 \\
        \hline
        0&1&1&1 \\
        \hline
        1&0&0&0 \\
        \hline
        1&0&1&0 \\
        \hline
        1&1&0&1 \\
        \hline
        1&1&1&0  \\
        \hline
\end{tabular}   
\caption{}
\label{tab:2005-gate-ec-54}
\end{table}
\item Minimize the logic function represented by the following Karnaugh map.
\label{prob:2010-gate-ee-52}
\hfill (CBSE 2021)
	\begin{karnaugh-map}[4][2][1][$YZ$][$X$]
		\manualterms{1,1,0,1,0,0,0,1}
	%	\implicant{0}{1}
	%	\implicant{3}{7}
	\end{karnaugh-map}	
\item Find the output for the Karnaugh map shown below
\label{tab:2019-gate-ec-34}
\hfill (GATE EE 2019)
	\begin{karnaugh-map}[4][4][1][$PQ$][$RS$]

		\minterms{1,3,4,5,7,6,12,13,15,14}
		\maxterms{0,2,8,9,10,11}

%		\implicant{4}{14}
%		\implicant{1}{7}
	\end{karnaugh-map}
\item The propogation delays of the XOR gate, AND gate and multiplexer (MUX) in the circuit shown in the Fig.
\ref{fig:2021-gate-ec-31}
	are 4 ns, 2 ns and 1 ns, respectively.
\label{prob:2021-gate-ec-31}
If all the inputs P, Q, R, S and T are applied simultaneously and held constant, the maximum propogation delay of the circuit is
\hfill (Gate EC-2021)
\begin{enumerate}
	\item 3 ns \item 5 ns \item 6 ns \item 7 ns
\end{enumerate}
\begin{figure}[!htb]
	\centering
\input{gate/2021/31/figs/diagram.tex}
	\caption{}
\label{fig:2021-gate-ec-31}
\end{figure}
\item 
\label{prob:2012-gate-ec-19}
Consider the 2-bit multiplexer(MUX) shown in the figure. For output to be the XOR of R and S, the values for $ W,X,Y$ and $Z$ are ? 
\hfill (GATE EC-2022)
\begin{figure}[h]
\input{gate/2022/19/figs/fig1.tex}
\caption{}
\label{fig:2012-gate-ec-19}
\end{figure}
\begin{enumerate}
\item $W = 0, X = 0, Y = 1, Z = 1$
\item $W = 1, X = 0, Y = 1, Z = 0$
\item $W = 0, X = 1, Y = 1, Z = 0$
\item $W = 1, X = 1, Y = 0, Z = 0$
\end{enumerate}
\item In the circuit shown below , X and Y are digital inputs, and Z is a digital output. The quivalent 
		circuit is a  
		\label{prob:2019 EE 36}
		\hfill(GATE EE 2019)
\begin{figure}[htbp]
\begin{center}
    \centering
    \begin{circuitikz}[scale=1]
    % 1st not
        \draw (0,0) node[not port,scale=0.8 ] (not) {};
        \draw (3,-0.28) node[and port] (and) {};
        
        \draw (not.in 1) -- ++(-1,0) node[left] {$X$};
        \draw (not.out) -- ++(0,0) coordinate (and.in 1);
        \draw (not.in 1) -- ++(0,-1.7) node[below] {$ $};
    % 1st And
        \draw (and.in 1) -- ++(-1.2,0) node[left] {$ $};
        \draw (and.in 2) -- ++(-3.3,0) node[left] {$Y$};
        \draw (and.out) -- ++(0,0) node[right] {$ $};

        \draw (and.in 2) -- ++(-3,0) coordinate (point);
        \draw (point) -- ++(0,-1.7) -- ++(1,0) node[below] {$ $};
        \draw (and.out) -- ++(0,-0.47) node[below] {$ $};
    %2nd not
        \draw (0,-2.28) node[not port ,scale=0.8] (not) {};
        \draw (not.in 1) -- ++(-0.8,0) node[left] {$ $};
        \draw (not.out) -- ++(0,0) coordinate (and.in 2);
   %2nd ANd
        \draw (3,-2) node[and port] (and) {};
        \draw (and.in 1) -- ++(-2.2,0) node[left] {$ $} ;
        \draw (and.in 2) -- ++(-1.2,0) node[left] {$ $} ;
        \draw (and.out) -- ++(0,0) node[right] {$ $};
          \draw (and.out) -- ++(0,0.7) node[above] {$ $};
    %or
        \draw (6,-1) node[or port] (or) {};
        \draw (or.in 1) -- ++(-1.48,0) node[left] {$ $} ;
        \draw (or.in 2) -- ++(-1.48,0) node[left] {$ $} ;
        \draw (or.out) -- ++(1,0) node[right] {$Z$};
     \end{circuitikz}
	\caption{ }
	\label{fig}
\end{center}
\end{figure}
    \begin{enumerate}
        \item NAND gate
        \item NOR gate
        \item XOR gate
        \item XNOR gate
    \end{enumerate}

\end{enumerate}
