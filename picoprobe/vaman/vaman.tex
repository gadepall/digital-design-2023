This section enumerates the steps needed to use the Picoprobe to debug an LC
Vaman board from  the command line using OpenOCD and GDB.

\subsection{Components}
\begin{table}[!ht]
    \input{picoprobe/vaman/tables/vaman-components.tex}
    \caption{List of components needed to debug Vaman with a Picoprobe.}
    \label{tab:vaman-comp}
\end{table}

\subsection{Wiring}
Wire the debugger to the other Raspberry Pi Pico board by following the
wiring diagram in \autoref{tab:vaman-wiring}.
\begin{table}[!ht]
    \input{picoprobe/vaman/tables/vaman-wiring.tex}
    \caption{Connections required to debug Vaman with a Picoprobe.}
    \label{tab:vaman-wiring}
\end{table}

\subsection{Building}
\begin{enumerate}
    \item Build a debuggable ELF file to load onto the Vaman by entering the
    following commands at a terminal window.
    \begin{lstlisting}
cd vaman/arm/codes/setup/blink/GCC_Project
make -j4
    \end{lstlisting}
    \item A successful build should generate the debuggable ELF file
    \begin{lstlisting}
vaman/arm/codes/setup/blink/GCC_Project/output/bin
    \end{lstlisting}
\end{enumerate}

\subsection{Debugging}
\begin{enumerate}
    \item Connect the debugger and Vaman to the laptop simultaneously.
    \item In a terminal window, start an OpenOCD server by entering the
    following commands.
    \begin{lstlisting}
cd pico/openocd
sudo src/openocd -f tcl/interface/cmsis-dap.cfg -f tcl/target/eos_s3.cfg -s tcl
    \end{lstlisting}
    A successful execution of these commands will result in the following
    (truncated) output.
    \begin{lstlisting}
Info: Listening on port 3333 for gdb connections
    \end{lstlisting}
    \item In \emph{another} terminal window, start \texttt{gdb} with the ELF
    file as follows.
    \begin{lstlisting}
cd vaman/arm/codes/setup/blink/GCC_Project/output/bin
gdb-multiarch blink.elf
(gdb) target extended-remote localhost:3333
(gdb) break main
(gdb) layout src
(gdb) continue
    \end{lstlisting}
    Here, you should see the source code for the program flashed onto the Vaman

    \emph{(Optional)} You can also use \texttt{tmux} instead of two separate
    terminal instances.
    \item You can execute instruction by instruction by typing \texttt{next} at
    the GDB prompt. Alternatively, you can step into functions called from the
    main function by \texttt{step}. 
    \item To reset to the start of the program and reach the main breakpoint
    again, type the following.
    \begin{lstlisting}
(gdb) monitor reset init
(gdb) load
(gdb) continue
    \end{lstlisting}
    To quit from gdb, type the following.
    \begin{lstlisting}
(gdb) quit
    \end{lstlisting}
    For other functionalities type \texttt{help} at the GDB prompt.
    \item To see serial output, attach a terminal to the debugger by typing the
    following commands at a terminal window.
    \begin{lstlisting}
sudo minicom -D /dev/ttyACM0 -b 115200
    \end{lstlisting}
    Note that if the debugger is not present at \texttt{/dev/ttyACM0}, then you
    can find the correct port by inspecting the output produced by the
    following command.
    \begin{lstlisting}
sudo dmesg -w
    \end{lstlisting}
\end{enumerate}