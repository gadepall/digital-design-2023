\iffalse
\documentclass[12pt]{article}
\usepackage{graphicx}
\usepackage[none]{hyphenat}
\usepackage[english]{babel}
\usepackage{caption}
\usepackage[parfill]{parskip}
\usepackage{hyperref}
\usepackage{import}
\usepackage{booktabs}
\usepackage{circuit}%%Importing Packing circiut.sty created for circuit
\def\inputGnumericTable{}
\usepackage{color}                                            
    \usepackage{array}                                            
    \usepackage{longtable}                                        
    \usepackage{calc}                                             
    \usepackage{multirow}                                         
    \usepackage{hhline}                                           
    \usepackage{ifthen}
\usepackage{array}
\usepackage{amsmath}  
\usepackage{parallel,enumitem}
\usepackage{listings}
\lstset{
language=tex,
frame=single,
breaklines=true
}
 
\begin{document}
\fi

\section{Software}
%
All codes used in this document are available in the following directory 
%निम्न जालबन्धन से इस लिख में उपयोग किए गए समस्त क्रमादेश अवाहरत करें।
\begin{lstlisting}
vaman/esp32/codes
\end{lstlisting}
\section{Flash Vaman-ESP32 using Arduino}

\renewcommand{\theequation}{\theenumi}
\renewcommand{\thefigure}{\theenumi}
\begin{enumerate}[label=\thesection.\arabic*.,ref=\thesection.\theenumi]
\numberwithin{equation}{enumi}
\numberwithin{figure}{enumi}
\numberwithin{table}{enumi}
\item Do not power any devices.  Make connections as shown in \tabref{tab:vaman/uart/rpi-vaman-uart} and 
\figref{fig:vaman/uart/1}. 
		The Vaman pin diagram is available in 
\figref{fig:vaman/fpga/setup/pin_sheet}
\begin{figure}
\centering
\begin{figure}[!ht]
\centering
\resizebox{1\textwidth}{!}{%
\begin{circuitikz}
\tikzstyle{every node}=[font=\Large]

\draw [, line width=0.9pt](4,6) to[short] (17,6);
\draw (4,6) to[short] (17,6);
\draw [, line width=0.9pt ] (5,7) rectangle (9,11);
\draw [, line width=0.9pt ] (15,7) rectangle (19,11);
\draw (11,10) to[short] (11.5,10);
\draw (11,9.5) to[short] (11.5,9.5);
\draw (11.5,10) node[ieeestd nand port, anchor=in 1, scale=0.89](port){} (port.out) to[short] (13.5,9.75);
\draw [, line width=0.9pt](11,9.5) to[short] (11,8.75);
\draw [, line width=0.9pt](9,10) to[short] (11,10);
\draw [, line width=0.9pt](13.5,9.75) to[short] (14,9.75);
\draw [, line width=0.9pt](14,9.75) to[short] (14,10);
\draw [, line width=0.9pt](14,10) to[short] (15,10);
\draw [, line width=0.9pt](19,10) to[short] (20.5,10);
\draw [, line width=0.9pt](20,10) to[short] (20,12);
\draw [, line width=0.9pt](20,12) to[short] (20,13);
\draw[, line width=0.9pt] (20,13) to[short] (10,13);
\draw [, line width=0.9pt](10,10) to[short] (10,12);
\draw (9,13) to[short] (8.75,13);
\draw (9,12.5) to[short] (8.75,12.5);
\draw (8.75,13) node[ieeestd xor port, anchor=in 2, scale=0.89, rotate=180](port){} (port.out) to[short] (6.75,12.75);
\draw[, line width=0.9pt] (10,13) to[short] (8.75,13);
\draw [, line width=0.9pt](10,12) to[short] (10,12.5);
\draw[, line width=0.9pt] (10,12.5) to[short] (9,12.5);
\draw[, line width=0.9pt] (6.75,12.75) to[short] (4,12.75);
\draw [, line width=0.9pt](4,12.75) to[short] (4,10);
\draw [, line width=0.9pt](4,10) to[short] (5,10);
\node [font=\Large] at (7,9) {A};
\node [font=\Large] at (17,9) {B};
\node [font=\Large] at (5.25,10) {D};
\node [font=\Large] at (15.25,10) {D};
\node [font=\Large] at (18.5,10) {Q};
\node [font=\Large] at (8.5,10) {Q};
\node [font=\Large] at (8.5,8) {$\bar Q$};
\node [font=\Large] at (18.5,8) {$\bar Q$};
\node [font=\Large] at (6,7.75) {CK};
\node [font=\Large] at (16,8) {CK};
\node [font=\Large] at (3.25,6) {CLK};
\node [font=\Large] at (11,8.5) {Xin};
\draw [, line width=0.9pt](6.75,6) to[short] (6.75,7);
\draw [, line width=0.9pt](17,6) to[short] (17,7);
\draw [line width=0.9pt, short] (6.25,7) -- (6.75,7.75);
\draw [line width=0.9pt, short] (6.75,7.75) -- (7.25,7);
\draw [line width=0.9pt, short] (16.5,7) -- (17,7.75);
\draw [line width=0.9pt, short] (17,7.75) -- (17.5,7);
\node [font=\Large] at (11.5,10.5) {$Q_A$};
\node [font=\Large] at (20.25,9.5) {$Q_B$};
\end{circuitikz}
}%

\label{fig:ide/fsm/figs/circuit}
\end{figure}

\caption{Circuit Connections}
\label{fig:vaman/uart/1}
\end{figure}
\begin{figure*}[!ht]
\centering
\includegraphics[width = \textwidth]{vaman/fpga/setup/figs/pin_sheet.png}
\caption{Pin diagram}
%\caption{कुश आरेख}
\label{fig:vaman/fpga/setup/pin_sheet}
\end{figure*}
			\begin{table}[!h]
		\input{vaman/uart/tables/rpi-vaman-uart.tex}
		\caption{}
		\label{tab:vaman/uart/rpi-vaman-uart}
	\end{table}
%\item Add the Potential Divider circuit between VAMAN and ARDUINO boards TX and RX pins as shown in Figure \ref{fig:vaman/uart/1}
%\item Now power on the devices and measure the voltage across the Vaman-ESP Rx pin using a multimeter.  Comment.
\iffalse
\item Modify your platformio.ini file by adding the lines
\begin{lstlisting}
upload_protocol = esptool
upload_port = /dev/ttyACM0
upload_speed = 115200
\end{lstlisting}
\fi
\item For compiling and genrating the bin file 
\begin{lstlisting}
cd vaman/esp32/codes/ide/blink
pio run
\end{lstlisting}
\item make sure that platformio.ini file contains these lines
\begin{lstlisting}
[env:esp32doit-devkit-v1]
platform = espressif32
board = esp32doit-devkit-v1
framework = arduino
platform_packages =                                                               toolchain-xtensa-esp32@https://github.com/esphome/esphome-docker-base/releases/download/v1.4.0/toolchain-xtensa32.tar.gz                             
framework-arduinoespressif32@<3.10006.210326
\end{lstlisting}
\item For uploading bin file to Vaman through ArduinoDroid application 
\begin{lstlisting}[language=bash]
1. Open the Droid Application
2. Click the three dots in the top right corner
3. Navigate to Settings -> Board Type
4. Select ESP32 -> DOIT ESP32 DEVKIT V1
5. Change the upload speed to 115200
6. Upload the generated .bin file
\end{lstlisting}
While the dots are printed on the screen, disconnect the EN wire from GND.   Make sure that the Vaman board is not powering any device while flashing.  The Vaman-ESP should now flash.
\iffalse
\item Execute the following
\begin{lstlisting}
cd vaman/esp32/codes/ide/blink
pio run 
pio run -t nobuild -t upload
\end{lstlisting}
While the dots and dashes are printed on the screen, disconnect the EN wire from GND.   Make sure that the Vaman board is not powering any device while flashing.  The Vaman-ESP should now flash.
\fi
\item After flashing, disconnect pin 0 on Vaman-ESP from GND. Power on Vaman and the appropriate LED will blink.
\item Open
\begin{lstlisting}
vaman/esp32/codes/ide/blink/src/main.cpp 
\end{lstlisting}
and change the delay to 
\begin{lstlisting}
delay(2000);
\end{lstlisting}
and exectute the code by following the steps above.
\end{enumerate}
